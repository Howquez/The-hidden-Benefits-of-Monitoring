\RequirePackage{filecontents}
\begin{filecontents}{Master_Thesis.bib}

@article{sebald2010GEB,
  title={Attribution and reciprocity},
  author={Sebald, Alexander},
  journal={Games and Economic Behavior},
  volume={68},
  number={1},
  pages={339--352},
  year={2010},
  publisher={Elsevier}
}

@article{dufwenberg2004GEB,
  title={A theory of sequential reciprocity},
  author={Dufwenberg, Martin and Kirchsteiger, Georg},
  journal={Games and Economic Behavior},
  volume={47},
  number={2},
  pages={268--298},
  year={2004},
  publisher={Elsevier}
}

@article{geanakoplos1989GEB,
  title={Psychological games and sequential rationality},
  author={Geanakoplos, John and Pearce, David and Stacchetti, Ennio},
  journal={Games and Economic Behavior},
  volume={1},
  number={1},
  pages={60--79},
  year={1989},
  publisher={Elsevier}
}

@article{dickinson2008GEB,
  title={Does monitoring decrease work effort?: The complementarity between agency and crowding-out theories},
  author={Dickinson, David and Villeval, Marie-Claire},
  journal={Games and Economic behavior},
  volume={63},
  number={1},
  pages={56--76},
  year={2008},
  publisher={Elsevier}
}

@article{Siemens2013JEBO,
  title={Intention-based reciprocity and the hidden costs of control},
  author={von Siemens, Ferdinand A},
  journal={Journal of Economic Behavior \& Organization},
  volume={92},
  pages={55--65},
  year={2013},
  publisher={Elsevier}
}

@article{hinkin2008JAP,
  title={An examination of "nonleadership": from laissez-faire leadership to leader reward omission and punishment omission.},
  author={Hinkin, Timothy R and Schriesheim, Chester A},
  journal={Journal of Applied Psychology},
  volume={93},
  number={6},
  pages={1234},
  year={2008},
  publisher={American Psychological Association}
}

\end{filecontents}


%\documentclass{article}
%\documentclass[openbib]{article}
\documentclass[leqno]{article}
%\documentclass[leqno,openbib]{article}
%%%%%%%%%%%%%%%%%%%%%%%%%%%%%%%%%%%%%%%%%%%%%%%%%%%%%%%%%%%%%%%%%%%%%%%%%%%%%%%%%%%%%%%%%%%%%%%%%%%%%%%%%%%%%%%%%%%%%%%%%%%%%%%%%%%%%%%%%%%%%%%%%%%%%%%%%%%%%%%%%%%%%%%%%%%%%%%%%%%%%%%%%%%%%%%%%%%%%%%%%%%%%%%%%%%%%%%%%%%%%%%%%%%%%%%%%%%%%%%%%%%%%%%%%%%%
\usepackage{indentfirst}
\usepackage{amsfonts}
\usepackage{ntheorem}
\usepackage{amsmath}
\usepackage[doublespacing]{setspace}
\usepackage{sectsty}
\usepackage[norule,bottom]{footmisc}
\usepackage[justification=centering,textfont={sc},labelfont={rm}]{caption}
\usepackage{varioref}
\usepackage{lipsum}
\usepackage{filecontents}
\usepackage{bibentry}
\usepackage{natbib}
\nobibliography*
\usepackage{fixltx2e}
\usepackage{xcolor}
\usepackage{amsmath}
\usepackage{hyperref}
\usepackage{hyperref} \hypersetup{colorlinks=true, allcolors=magenta}
\theoremseparator{:}
\newtheorem{hyp}{Hypothesis}
\usepackage{tikz}
\usetikzlibrary{positioning}
\usepackage[latin1]{inputenc} 
\usepackage[english]{babel}
\usepackage{graphicx}
\newcommand{\indep}{\rotatebox[origin=c]{90}{$\models$}}
\usepackage{xfrac}










%TCIDATA{CSTFile=QJEconomics.cst}

\definecolor{dark-gray}{gray}{0.15}
\makeatletter
\newcommand{\globalcolor}[1]{%
  \color{#1}\global\let\default@color\current@color
}
\makeatother

\AtBeginDocument{\globalcolor{dark-gray}}



\theoremindent\parindent
\makeatletter    
\renewtheoremstyle{plain}{\item[\hskip\labelsep\hskip-\parindent \theorem@headerfont ##1\ ##2\theorem@separator]}{\item[\hskip\labelsep\hskip-\parindent \theorem@headerfont ##1\ ##2\ (##3)\theorem@separator]}
\makeatother
\theoremheaderfont{\scshape}
\theorembodyfont{\rmfamily}
\theoremseparator{.}
\newtheorem{theorem}{Theorem}
\newtheorem{acknowledgement}[theorem]{Acknowledgement}
\newtheorem{algorithm}[theorem]{Algorithm}
\newtheorem{axiom}[theorem]{Axiom}
\newtheorem{case}[theorem]{Case}
\newtheorem{claim}[theorem]{Claim}
\newtheorem{conclusion}[theorem]{Conclusion}
\newtheorem{condition}[theorem]{Condition}
\newtheorem{conjecture}[theorem]{Conjecture}
\newtheorem{corollary}[theorem]{Corollary}
\newtheorem{criterion}[theorem]{Criterion}
\newtheorem{definition}[theorem]{Definition}
\newtheorem{example}[theorem]{Example}
\newtheorem{exercise}[theorem]{Exercise}
\newtheorem{lemma}[theorem]{Lemma}
\newtheorem{notation}[theorem]{Notation}
\newtheorem{problem}[theorem]{Problem}
\newtheorem{proposition}[theorem]{Proposition}
\newtheorem{remark}[theorem]{Remark}
\newtheorem{solution}[theorem]{Solution}
\newtheorem{summary}[theorem]{Summary}
\newenvironment{proof}[1][Proof]{\indent\textit{#1.}}{\ \rule{0.5em}{0.5em}}
\sectionfont{\normalfont\scshape\centering}
\renewcommand{\thesection}{\Roman{section}.}
\renewcommand{\thesubsection}{\thesection\Alph{subsection}.}
\subsectionfont{\itshape}
\makeatletter
\def\@biblabel#1{\hspace*{-\labelsep}}
\renewenvironment{thebibliography}[1]
 {\section*{\refname} \@mkboth{\MakeUppercase\refname}{\MakeUppercase\refname} \list{\@biblabel{\@arabic\c@enumiv}} {\singlespacing
 \settowidth\labelwidth{\@biblabel{#1}} \setlength{\itemsep}{0pt plus 1pt} \leftmargin\labelwidth
 \advance\leftmargin\labelsep
 \@openbib@code
 \usecounter{enumiv} \let\p@enumiv\@empty
 \renewcommand\theenumiv{\@arabic\c@enumiv}} \sloppy
 \clubpenalty4000
 \@clubpenalty \clubpenalty
 \widowpenalty4000 \sfcode`\.\@m}
 {\def\@noitemerr
 {\@latex@warning{Empty `thebibliography' environment}} \endlist}
\makeatother
\makeatletter
\renewcommand\@makefnmark{\mbox{\textsuperscript{\normalfont\@thefnmark}}}
\renewcommand\@makefntext[1]{\indent\makebox[2.5em][r]{\@thefnmark.\,}#1}
\if@titlepage
  \renewenvironment{abstract}{      \titlepage
      \null\vfil
      \@beginparpenalty\@lowpenalty
      \begin{center}        \@endparpenalty\@M
      \end{center}}     {\par\vfil\null\endtitlepage}
\else
  \renewenvironment{abstract}{      \if@twocolumn
      \else
        \small
        \begin{center}        \end{center}      \fi}
      {\if@twocolumn\else\endquotation\fi}
\fi
\renewcommand\thetable{\@Roman\c@table}
\renewcommand\tablename{TABLE}
\renewcommand\thefigure{\@Roman\c@figure}
\renewcommand\figurename{\textsc{Figure}}
\makeatother
\labelformat{equation}{(#1)}

\begin{document}

\title{The (hidden) benefits of managerial attention}
\author{%
\textsc{Hauke Roggenkamp}}
\maketitle



%TCIMACRO{%
%\TeXButton{Titlepage formatting}{\thispagestyle{empty}\setcounter{page}{0}}}%
%BeginExpansion
\thispagestyle{empty}\setcounter{page}{0}%
%EndExpansion













%%%%%%%%%%%%%%%%%%%%%%%%%%%%%%%%%%%%%%%%%%%%%%%%%%%%%%%%%%%%%%%%%%%%%%%%%%%%%%%%%%%%%%%%%%%%%%%%%
%%%%%%%%%%%%%%%%%%%%%%%%%%%%%%%%%%%%%%%%%%%%%%%%%%%%%%%%%%%%%%%%%%%%%%%%%%%%%%%%%%%%%%%%%%%%%%%%%
%%%%%%%%%%%%%%%%%%%%%%%%%%%%%%%%%%%%%%%%%%%%%%%%%%%%%%%%%%%%%%%%%%%%%%%%%%%%%%%%%%%%%%%%%%%%%%%%%
\newpage
\tableofcontents
\thispagestyle{empty}\setcounter{page}{0}%

\newpage
\section{Motivation}
Attention is an essential tool of (human resource) management since it allows managers to better observe the employees' effort provision. In classical principal-agent-settings where the actual performance does not perfectly reflect the employees' effort, managerial attention can thus be helpful to reward or punish employees, for instance, in wage (re)negotiations, promotions or dismissals. Besides the obvious (opportunity) costs of the principal, there may, however, be implicit costs and benefits attached, which might render its value ambiguous. That is to say that  attention as a management tool might have varying latent effects. If managers take this into account, they might end up observing only those employees who benefit from attention and who therefore want to be observed. 

In very broad terms, we want to investigate whether there is variation in the response to managerial attention and whether this variation can be explained by the employees' productivity. On top of that, we are interested in the managers' anticipation of these subtile differences, that is, we want to test whether and how they account for them.




%%%%%%%%%%%%%%%%%%%%%%%%%%%%%%%%%%%%%%%%%%%%%%%%%%%%%%%%%%%%%%%%%%%%%%%%%%%%%%%%%%%%%%%%%%%%%%%%%
\section{The Predecessor}

\begin{figure}
	\includegraphics[trim = 15mm 35mm 15mm 35mm, clip,scale=0.7]{20170406_GameOldTree}
    \caption{The Predecessor 2nd Stage}
    \label{fig:oldstage2}
\end{figure}

The experiment consists of two stages as well as two players, a principal and an agent. In the first stage, both players face a real effort task that affects the individual's own payments. The second stage matches principals with agents and prompts principals to make a decision whereas agents are asked to exert effort. Both actions, that is, the principal's decision and the agent's effort provision, affect the payment of both players interdependently. The following paragraphs describe the actions as well as the payment structure in more detail.

The real effort task in stage 2 is the same as in stage 1. The only difference is that both principals \emph{and} agents exert effort in stage 1 whereas \emph{only} the agents are asked to exert effort in stage 2. The real-effort task thereby looks as follows: Subjects face a fixed number of screens, each displaying a number of randomly arranged boxes. While subjects have to click on as many boxes as possible, which causes them to vanish, a timer is running such that a new screen with new boxes will appear, usually before all boxes of the previous screen are "clicked away".

The second stage, as illustrated in Figure \ref{fig:oldstage2}, begins with player 1's continuous decision about her information technology, denoted as IT. This decision is costly and increases in IT | a higher IT costs more than a low one. These costs are denoted by $c_{IT}$ in player 1's payoffs. By choosing between \textit{High} or \textit{Low}, player 1 essentially chooses the probability distribution of player $0$'s explicit randomization (such that the probability of the effort dependent mechanism being pivotal is p(IT)). The decision is followed by player 2's effort provision, $e \in (0,1)$. After both players made their decisions or completed the real-effort task , a third player, Chance (denoted by $0$), explicitly randomizes the players' payment mechanism, which either depends on player 2's effort provision or is determined randomly (over q and (1-q)). Afterwards, player $0$ randomly sets the other players' payoffs subject to the particular probability distribution given by either $e$ or $q$. 

The payoffs can be explained as follows: Player 1 always receives the piece rate of player 2's effort provision, $e$. This payoff is reduced by the costs of player 1's IT-decision. Player 2 always receives a fixed wage, $w>0$, and might receive a bonus payment, $b>0$. Notably, the design suggested in Figure \ref{fig:stage2} directly charges player 1 for the bonus payment, $b$, player 2 might receive.




%%%%%%%%%%%%%%%%%%%%%%%%%%%%%%%%%%%%%%%%%%%%%%%%%%%%%%%%%%%%%%%%%%%%%%%%%%%%%%%%%%%%%%%%%%%%%%%%%
\section{Adjustments}
The experimental design described above is difficult to understand and might suffer experimenter demand effects in the sense that subjects who show up to experiments may be reluctant to shirk (that is, to exert a low level of effort) if the experimenter is in the same room. In addition, participants might already have the expectation to work if they show up or might lack entertaining alternatives to the effort task (making shirking boring). Another concern is that agents do not incorporate the principal's investment decision into their reactions because it is not salient enough at the point of time where they provide effort. These issues turn out to be a problem because we want to observe variation in the effort provision and hope to be able to explain it with different levels of information technology (IT) or '\emph{managerial attention}' and productivity (performance measured in stage 1). 

We therefore think about adjusting the design in at least three different ways:
\begin{enumerate}
      \item The principal's decision will be binary and therefore easier to understand for both players.
      %\item The experiment will be conducted online to reduce experimenter demand effects and to make effort provision less attractive due to an increment of shirking opportunities.
      \item The agent will get the possibility to make a deliberate decision about her effort provision 
      before she exerts effort. That is, she will commit to a maximum of effort she will provide as a direct 
      response to the principal's investment decision.
      \item To observe more variation, we exogenously influence the agent's effort provision by 
      introducing a treatment in which we manipulate the time each screen is displayed in the real-effort 
      tasks. Such a manipulation would affect the task's difficulty level and therefore also the players' 
      effort provision. 
\end{enumerate}


%We can furthermore consider to change two other features of the experiment:\begin{itemize}\item The principal's payoff function or a second treatment with an additional function: As I see it, we have at least two options: Either a piece rate or a set of binary wages where the probability of receiving the high wage increases in the agent's effort. While the former offers a more direct channel of reciprocity, the second has the advantage that it is impossible to make inferences about the agent's effort provision. More precisely, the agent knows that the principal to find out how much effort she provided.\item Incentivized control questions: We can think about a maximum number of attempts to answer the control questions correctly or an additional payment (or fine) if the answers were (not) answered correctly immediately. Since the online subject pool is large we can think about excluding subjects who are not able to find the correct answers with fewer than 3 attempts, we exclude her. Given that it is an online experiment, which implies that people have no effort to participate and that our payoff determination (and therefore the whole experiment) is hard to understand, at least a count of attempts might turn out to be useful for the analysis.\end{itemize}




%%%%%%%%%%%%%%%%%%%%%%%%%%%%%%%%%%%%%%%%%%%%%%%%%%%%%%%%%%%%%%%%%%%%%%%%%%%%%%%%%%%%%%%%%%%%%%%%%
\section{New Design}
The experiment consists of two stages. The first stage basically consists of an individual real-effort task\footnote{Subjects face a fixed number of screens, each displaying a number of randomly arranged boxes. While subjects have to click on as many boxes as possible, which causes them to vanish, a timer is running such that a new screen with new boxes will appear, usually before all boxes of the previous screen are "clicked away".}, which we use to elicit the subjects' productivity $e^0_i$. At the end of the stage, both players, 1 and 2, will learn about their own and 2's productivity. %I assume that both players believe that player 2's productivity|and therefore her costs of effort provision $c(e)$|is constant over time. That is, given the same incentives, both players expect her to provide a similar level of effort, that is  $e^0_2 \pm\eta_2 \text{ with } \eta_2 \sim \mathcal{N}(0,\,\sigma^{2})$.

The second stage, illustrated in Figure \ref{fig:stage2}, includes the same real-effort task but includes additional actions and players: At history $h^0$, player 1 faces a \emph{binary} decision and chooses an IT which is either low or high.\footnote{Remember that player 1 and 2 know $e^0_2$ and that Player 2 knows that player 1 knows it.} 1's IT decision is costly with $c_H>c_L\geq0$. 
Subsequently, player 2 decides on $n$, which is the number of screens she intends to work on. This decision is followed by her effort provision, $e \in (0,1)$, with a ceiling determined by her choice of $n \in (0,n_{max})$. After both players made their decisions, a third player, Chance (denoted by $0$), explicitly randomizes the players' payment mechanism, which either depends on player 2's effort provision or is determined randomly (over q and (1-q)). Afterwards, player $0$ randomly sets the other players' payoffs subject to the particular probability distribution given by either $e$ or $q$. Note that, by choosing either \textit{High} or \textit{Low}, Player 1 essentially chooses the probability distribution of player $0$'s explicit randomization (such that the probability of the effort dependent mechanism being pivotal is either p or (1-p)). The payoffs' intuition is the following: Player 1 always receives the piece rate of player 2's effort provision, $e$. Following $h^1$ ($h^2$) she has to pay the costs of the high IT, $c_L$ ($c_H$), leading to a payoff of $e-c_L$ ($e-c_H$) at each end node of the particular subgame. Player 2 always receives a fixed wage, $w>0$, and might receive a bonus payment, $b>0$. Notably, the design suggested in Figure \ref{fig:stage2} does not directly charge player 1 for the bonus payment, $b$, player 2 might receive.

\begin{figure}
	\includegraphics[trim = 15mm 35mm 15mm 35mm, clip,scale=0.7]{20170406_GameTree}
    \caption{Stage 2}
    \label{fig:stage2}
\end{figure}





%%%%%%%%%%%%%%%%%%%%%%%%%%%%%%%%%%%%%%%%%%%%%%%%%%%%%%%%%%%%%%%%%%%%%%%%%%%%%%%%%%%%%%%%%%%%%%%%%
\section{Outline of The Workhorse Model}
\textbf{Note that you can skip this part since I am relying on your models, i.e.} \citep{sebald2010GEB, dufwenberg2004GEB}.\\\\
Following \cite{sebald2010GEB} and \cite{dufwenberg2004GEB}, I presume the subjects' to care about others' intentions such that they reciprocate kind with kind and unkind with unkind behavior in a sequential game. Because the underlying experiment involves two explicit randomizations, I employ  the model laid down in \cite{sebald2010GEB} to analyze the subjects' strategies. That is to say that I assume the subjects' utility to depend on their own material payoff \(\pi_i\) and a psychological payoff which can be described as the interaction of \(Y_{ij}\), \(\kappa_{ij}\) and \(\lambda_{iji}\) representing player i's sensitivity to (un)kindness of player j, player i's kindness towards player j and player i's perception of how kind player j treats player i respectively:\footnote{Conceptions of histories (h), i's strategy ($a_i$), i's belief about j's strategy ($b_{ij}$), i's belief about j's belief about k's strategy ($c_{ijk}$) as well as the strategy of the player chance ($\omega$) are borrowed from \cite{geanakoplos1989GEB, dufwenberg2004GEB, sebald2010GEB} and not further explained in this note.}

\vspace{-10mm}
\begin{multline} \label{eq:1}
	U_i(a_i(h),(b_{ij}(h))_{j \neq i}, (c_{ijk}(h))_{k \neq j}, \omega) = 
	\pi_i(a_i(h),(b_{ij}(h))_{j \neq i}, \omega)\\
	+ \sum_{j \neq i} Y_{ij} \cdot \kappa_{ij}(a_i(h),(b_{ij}(h))_{j \neq i}, \omega)  
	\cdot \lambda_{ijk}((b_{ij}(h))_{j \neq i}, (c_{ijk}(h))_{k \neq j}, \omega)
\end{multline}
The kindness of player i to another player j$\neq$i at any history h is defined as:

\vspace{-10mm}
\begin{multline*}
	\kappa_{ij}(a_i(h),(b_{ij}(h))_{j \neq i}, \omega) = 
	\pi_j(a_i(h),(b_{ij}(h))_{j \neq i},\omega) 
	- \pi^{ei}_j((b_{ij}(h))_{j \neq i}, \omega)
\end{multline*}
Which means that the kindness of i towards j in history h is modeled as the difference between the expected material payoff of j, $\pi_j$, that i intends to give j given i's belief about j's strategy and a term $\pi^{ei}_j$ which is defined as

\vspace{-10mm}
\begin{multline*}
	\pi^{ei}_j((b_{ij}(h))_{j \neq i}, \omega) = 
	\sfrac{1}{2} [max\{\pi_j(a_i(h), (b_{ij}(h))_{j \neq i}, \omega | a_i(h) \in A_i\}\\
	+ min \{\pi_j(a_i(h), (b_{ij}(h))_{j \neq i}, \omega | a_i(h) \in \xi_i\}]
\end{multline*}
where $\xi_i$ is a set of efficient strategies as explained in \cite{dufwenberg2004GEB}. In words, $\pi^{ei}_j((b_{ij}(h))_{j \neq i}, \omega)$ is used as a benchmark to evaluate i's kindness towards j and describes an equitable payoff for j given i's belief about j's strategy and the strategy of \emph{chance}. 

Using a similar term that describes i's equitable material payoff

\vspace{-10mm}
\begin{multline*}
	\pi^{ej}_i((c_{iji}(h))_{j \neq i}, \omega) = 
	\sfrac{1}{2} [max\{\pi_i(a_j(h), (c_{iji}(h))_{j \neq i}, \omega)| a_i(h) \in A_i\}\\
	+ min \{\pi_i(a_j(h), (c_{iji}(h))_{j \neq i}, \omega) | a_i(h) \in A_i\}],
\end{multline*}
i's \emph{belief} about how kind j is to her at history h is defined as:

\vspace{-10mm}
\begin{multline*}
	\lambda_{iji}((b_{ij}(h))_{j \neq i}, (c_{iji}(h))_{k \neq j}, \omega) =
	\pi_i(b_{ij}(h),(c_{iji}(h))_{k \neq h}, \omega)
	- \pi^{ej}_i((c_{iji}(h))_{j \neq i}, \omega) 
\end{multline*}




%%%%%%%%%%%%%%%%%%%%%%%%%%%%%%%%%%%%%%%%%%%%%%%%%%%%%%%%%%%%%%%%%%%%%%%%%%%%%%%%%%%%%%%%%%%%%%%%%
\section{Formal Solution}
Because player 1 (the principal) and 2 (the agent) know that player 0 (chance), who is explicitly randomizing, has the last move, they calculate their profits in terms of expected payoffs. Assuming that 2 believes that 1 expects 2 to replicate her effort from the first stage $e^0_2$, the equitable payoff that player 1 can give player 2 is  determined by:

\vspace{-10mm}\begin{align*}
	\pi^{e1}_2 (c_{212}(h), \omega ) & = \sfrac{1}{2} \cdot \{p[e^0_2(w+b)+(1-e^0_2)w]+(1-p)[q(w+b)+(1-q)w]\}\\
	& + \sfrac{1}{2} \cdot \{(1-p)[e^0_2(w+b)+(1-e^0_2)w]+p[q(w+b)+(1-q)w]\}\\
	& = \sfrac{1}{2} \cdot \{w+b[pe^0_2+(1-p)q] +w+b[(1-p)e^0_2+p\cdot q]\}\\
	& = w + \sfrac{1}{2} \cdot b(e^0_2+q)
\end{align*}
From here it follows that

\vspace{-10mm}\begin{align*}
	\lambda_{212}(h^1) & = w+b[pe^0_2+(1-p)q] - w - b(e^0_2+q)\\
	& = b[e^0_2(p-\sfrac{1}{2})+q(1-p-\sfrac{1}{2})],
\end{align*}
and similarly 

\vspace{-10mm}\begin{align*}
	\lambda_{212}(h^2) & = w+b[(1-p)e^0_2+p)q] - w - b(e^0_2+q)\\
	& = b[e^0_2(1-p-\sfrac{1}{2})+q(p-\sfrac{1}{2})].
\end{align*}
Which implies that both decisions, \textit{Low} and \textit{High}, can be perceived as kind or unkind|depending on the initial effort provision of player 2. If for instance, $p\equiv\sfrac{1}{4}$, $q\equiv\sfrac{1}{2}$ and $e^0_2<\sfrac{1}{2}$, $\lambda_{212}(h^1) \text{ will be positive, while } \lambda_{212}(h^2)$ will be negative. The exact opposite applies to the case where $e^0_2>\sfrac{1}{2}$, while it is perceived as neutral, that is, $\lambda_{212}(h^2)=0$ if $e^0_2=\sfrac{1}{2}$.

According to player 2's maximization problem stated in equation \ref{eq:1}, she maximizes her psychological payoff by reciprocating perceived (un)kindness with (un)kindness if $Y_{21}$ is positive. Since player 1's payoff heavily depends on player 1's effort provision, it is most intuitive to think of 1's equitable payoff as 2's initial effort provision, that is $\pi^{e2}_1\equiv e^0_2$.\footnote{Note that this is an assumption. Alternatively, one could argue in the spirit of \cite{dufwenberg2004GEB, sebald2010GEB} assuming $e^0_2$ to be the maximum effort, player 1 can provide such that $\pi^{e2}_1\equiv \frac{e^0_2-0}{2}$.} Her kindness is therefore expressed by:

\vspace{-10mm}
\begin{align*}
	\kappa_{21} & = e - e^0_2 
\end{align*}
Neglecting the material payoff for a moment, this essentially means that player 2 tries to increase her effort in stage 2 if she wants to be kind to player 1 and decreases her effort, for instance by pre-commiting to a ceiling by choosing n, if she intends to treat player 1 unkindly.

Putting the pieces from equation \ref{eq:1} together with player 2's costs of effort provision, $c(e)$, her payoff at history $h^1$ is given by:

\vspace{-10mm}
\begin{multline*}
	U_2(e|h^1) = \overbrace{w+b\ [pe+(1-p)\ q]}^{\text{material payoff}}\text{ } 
	+ \overbrace{Y_{21}}^{\text{sensitivity}}
	\cdot\ \overbrace{(e - e^0_2)}^{\text{kindness}}\\ 
	\cdot\ \underbrace{b\ [e^0_2(p-\sfrac{1}{2})+q\ (1-p-\sfrac{1}{2})]}_{\text{perception of kindness}}-c(e)
\end{multline*}

Maximizing $U_2(e|h^1)$ over $e$ yields:

\vspace{-10mm}
\begin{align*}
	c'(e) & = b\ p + Y_{21}\ b\ [e^0_2(p-\sfrac{1}{2})+q\ (1-p-\sfrac{1}{2})]\\
	& = b\ p + Y_{21}\ b\ [e^0_2(p-\sfrac{1}{2})+q\ (\sfrac{1}{2}-p)]\\
	c'(e) & = b\ p + Y_{21}\ b\ (\sfrac{1}{2}-p)(q-e^0_2)
\end{align*}
with $e^{o}\leftrightarrow c'(e^{o}) = b\ p$ being the optimal effort provision at $h^1$ if player 2 was \emph{not} reciprocal, that is, if $Y_{21}=0$. Intuitively, this equation states that player 2 will provide the optimal amount of effort $e^{o}$ plus a term that can be either positive or negative. Given that $p, q\leq\sfrac{1}{2}$ and that $Y_{21}>0$ this term will be negative if $e^0_2>q$ and positive if $e^0_2<q$. Assuming $c'(e)$ to be an increasing function, an agent being observed by the low IT would therefore exert an effort level

\vspace{-10mm}
\begin{align*}
	e^*(e^0_2>q, Low) & =c^{-1}_e\Big( b\ p + Y_{21}\ b\ (\sfrac{1}{2}-p)(q-e^0_2)\Big)<e^{o}\\
	e^*(e^0_2<q, Low) & =c^{-1}_e\Big( b\ p + Y_{21}\ b\ (\sfrac{1}{2}-p)(q-e^0_2)\Big)>e^{o}
\end{align*}
where $e^*(\cdot)$ is the level of effort that solves player 2's maximization problem and $c^{-1}_e(\cdot)$ is the inverse of the marginal costs function. Similarly, one can derive $U_2(e|h^1)$ and the resulting effort levels:

\vspace{-10mm}
\begin{align*}
	U_2(e|h^2) & = w+b[(1-p)e+p\ q] + Y_{21}(e - e^0_2)\\ 
	& \qquad\cdot b\ [e^0_2(1-p-\sfrac{1}{2})+q\ (p-\sfrac{1}{2})]-c(e)\\
	c'(e) & =  b\ p + Y_{21}\ b\ (\sfrac{1}{2}-p)(e^0_2-q)\\
	e^*(e^0_2>q, High) & =c^{-1}_e\Big( b\ p + Y_{21}\ b\ (\sfrac{1}{2}-p)(e^0_2-q)\Big)>e^{o}\\
	e^*(e^0_2<q, High) & =c^{-1}_e\Big( b\ p + Y_{21}\ b\ (\sfrac{1}{2}-p)(e^0_2-q)\Big)<e^{o}
\end{align*}
Because $e^{o}$ is also the solution to the first stage's maximization problem, which simply is $max_e[\pi_i(e)= w+b\ p\ e - c(e)]$ with $p=1$, $e^{o}=e^0_2$ such that the equations above implicitly compare $e^*$ with $e^0_2$. STOP: I FORGOT THAT THE MAXIMIZATION IS DIFFERENT, SINCE p NEEDS TO BE INCLUDED IN THE SECOND STAGE. (Un)kind choices therefore lead to a reduction (an increase) of player 2's effort provision in the second stage compared to the first one as long as she has a sensitivity for reciprocity, $Y_{21}>0$.

Comparing the case where player 2 has such preferences with the one where she has no sensitivity for reciprocity, it is now easy to see that a non-reciprocal player always increases her effort in response to an increase in $p$, that is, an investment into IT or '\emph{managerial attention}', while the effect of such an investment is ambiguous if player 1 faces a reciprocal opponent. In other words, the absence of reciprocity concerns, performance always increases in managerial attention | this might, however, not hold in presence a reciprocal player 2.

Assuming reciprocity concerns, a rational first player will condition her decision on more than just the IT's costs. Player 1's choice depends on which of the actions give her the higher expected utility given player 2's publicly known initial productivity. That is to say that she will choose \textit{High} if 

\vspace{-10mm}
\begin{align*}
	e^*(e^0_2, High)-c_H &> e^*(e^0_2, Low)-c_L \\
	\leftrightarrow e^*(e^0_2, High)-e^*(e^0_2, Low) &> c_H-c_L,
\end{align*}
and \textit{Low} otherwise.\footnote{Note that player 1's costs of IT have been neglected in player 2's considerations. See equation \ref{eq:kappa} for one approach to incorporate them.}




%%%%%%%%%%%%%%%%%%%%%%%%%%%%%%%%%%%%%%%%%%%%%%%%%%%%%%%%%%%%%%%%%%%%%%%%%%%%%%%%%%%%%%%%%%%%%%%%%
\section{Research Questions}
\begin{enumerate}
      \item Does managerial attention (choosing the high level of IT) always increase performance?
      \item Do principals practicing attention earn more money?
      \item Does a higher difficulty of the real-effort task (leading to lower productivity) affect the
       reaction to managerial attention? That is, do more people reduce their effort as a response to 
       managerial attention under a high difficulty treatment compared to a treatment with a low level of 
       difficulty?
      \item Or do principals (correctly) anticipate reciprocity concerns and increase their payoffs by 
      tailoring their managerial attention to the agents' productivities? 
      
\end{enumerate}




%%%%%%%%%%%%%%%%%%%%%%%%%%%%%%%%%%%%%%%%%%%%%%%%%%%%%%%%%%%%%%%%%%%%%%%%%%%%%%%%%%%%%%%%%%%%%%%%%
\newpage\section{Open Questions}
\begin{enumerate}
      \item How shall we model perceived kindness if the principal does not know the agent's productivity?
      \item What if the principal didn't even knew her productivity in stage 1 until the end of the
      experiment?
      \item If the principal had to pay the agent's bonus, the agent would have two options to harm or favor
      the principal. If she wanted to harm the principal, she could either increase her effort to reduce the
      principal's payment (due to the bonus) or decrease her effort to generate a low output in the first
      place.
      \item How should the principal's payment function look like with respect to the piece rate and a flat
      wage? The previous design implemented a relatively high flat wage combined with a low piece rate
      \((e \cdot 0.25)\). However, a low piece rate decreases the effectiveness of the agent's reciprocity.
      As a consequence of such a low piece rate, the agent has to decrease her expected utility considerably
      to reciprocate kind or unkind behavior. Reciprocity is therefore less likely but detected easily. The
      opposite holds for high piece rates since unintentional, small changes in effort provision might be 
      interpreted as reciprocity.
      \item Shall we really randomize which stage is payed out? This would mean that the agent faces 3 
      explicit randomizations in total. Isn't that too much?
      \item Is the assumption of c(e) reasonable? What if subjects find the task enjoyable. How would the
      maximization problem look like if the costs were non-positive?
      \item Can and shall we address crowding out effects due to intrinsic motivation? To put it differently,
      is it worth the effort? Since the task does not change between the stages, the initial intrinsic 
      motivation held constant. However, the material incentives, which may decrease intrinsic motivation
      change between both stages. If there was intrinsic motivation attached to the task, a reduction in 
      effort might be a reasonable response to an increase of managerial attention. If more productive agents
      exhibit a higher intrinsic motivation, their response to attention might differ from unproductive
      agents such that it would blur our analysis.
      \item Is there any questionnaire that would make sense to attach?
      	\begin{itemize}
			\item \href{http://www.mindgarden.com/documents/MLQGermanPsychometric.pdf}{Multifactor Leadership
			Questionnaire} see, for instance cite{hinkin2008JAP} for an analysis of \textit{laissez-faire
			leadership}.
			\item \href {http://www.sciencedirect.com/science/article/pii/S0749597805001184}{Leader Reward
			and Punishment Questionnaire}
			\item Big Five
			\item Rosenberg self-esteem scale
		\end{itemize}
	\item Is there any possibility to identify reciprocity concerns on an individual level?
      
      
\end{enumerate}



%%%%%%%%%%%%%%%%%%%%%%%%%%%%%%%%%%%%%%%%%%%%%%%%%%%%%%%%%%%%%%%%%%%%%%%%%%%%%%%%%%%%%%%%%%%%%%%%%
%\section{Quantities of Interest}
%The experiment allows us to observe each subject's \texttt{productivity} of the box-clicking task in stage 1, the principal's choice of IT, \texttt{info}, the agent's effort ceiling, \texttt{nscreens}, as well as her \texttt{effort} provided in stage 2. Note that both, \texttt{productivity} and \texttt{effort} \(\in [0,1]\).  Given the exogenous variable \(p_r=0.5\) as well as the \texttt{info} dummy, we can define the likely perception of \texttt{kindness} as 
%\[ \equiv
%  \left.
%  	\begin{cases}
%    	-1  & \quad \text{if } p_r < \text{\texttt{productivity}}\\
%    	1       & \quad \text{if } p_r \geq \text{\texttt{productivity}}\\
%  	\end{cases} 
%  \right|info=0
%\] and 
%\[ \equiv
%  \left.
%  	\begin{cases}
%    	-1  & \quad \text{if } p_r \geq \text{\texttt{productivity}} \\
%    	1       & \quad \text{if } p_r < \text{\texttt{productivity}}\\
%  	\end{cases} 
%  \right|info=1
%\]
%which is explained easily. Assume that an agent chooses not to invest in information technology, that is, \texttt{info}=0 and \(q\equiv prob(m=m_e)= q_l\), which means that the chance that the agent's payoff is determined by chance, \(p_r=1/2\), is relatively high. A non-na"ive agent who has demonstrated a \texttt{productivity}\(\leq p_r\) in stage 1 can expect a higher payoff under the random mechanism than under the effort dependent mechanism. She should therefore be more likely to perceive the principal's low investment decision as kind while a productive agent (\texttt{productivity}\(> p_r\)) expects to earn less under the random mechanism and should be more likely to expect the principal's low investment decision as unkind. These assumptions are formalized in the first bracket while the second one presents the intuition of the opposite case, where the agent decides to invest into information technology.
%
%Given our hypotheses, we expect this perception of \texttt{kindness} to explain some of the variance in our outcome variables. Our most important outcome variable, besides \texttt{nscreens}, does not regard absolute levels but differences in effort provision:\footnote{Since it is only the agent who has the opportunity to exert effort twice, I'll henceforth omit the subscript.}  
%
%\[\Delta\text{\texttt{effort}}_A\equiv \frac{\text{\texttt{effort}}_A - \text{\texttt{productivity}}_A}{\text{\texttt{productivity}}_A}\]
%
%Our first hypothesis basically states that, this difference is lower for those who perceive the principal's investment decision as unkind than for those who perceive their partner's decision as kind.  We would therefore reject our hypothesis if we do not find such a \emph{difference in differences} that can be explained by \texttt{kindness}. But if we observed it, our most favored explanation for such an observation is that the agent balances the material loss of a decreased effort with the psychological utility of harming the principal by providing less effort.\footnote{Remember that the principal's payoff depends on the agent's effort.} 
%
%On top of that, we are interested in the \emph{principal's} consideration of these reciprocity concerns. In the light of the perception of her investment in information technology the principal's benefit of this investment becomes ambiguous. Facing a productive agent, a principal should expect hidden benefits if she invested in information technology due to reciprocity, that is, if anything, many agents should rather increase than decrease their effort provision, which leads to a higher output and thus to a higher expected payoff for the principal. Again, the intuition for unproductive agents is exactly the opposite. Our second hypothesis would consequently be rejected if we did not observe a higher (lower) fraction of principals investing in technology if they face a productive (unproductive) agent.
%
%
%
%
%%%%%%%%%%%%%%%%%%%%%%%%%%%%%%%%%%%%%%%%%%%%%%%%%%%%%%%%%%%%%%%%%%%%%%%%%%%%%%%%%%%%%%%%%%%%%%%%%%
%\section{Analysis}
%With \(\text{\texttt{kindness}}_{ji}\) as a function of the agent's \(\text{\texttt{productivity}}_i\), the principal's \(\text{\texttt{info}}_j\) and the exogenously given \(p_r\), as well as possible covariates \(X_i\), we can run the following \emph{short regressions}\footnote{In contrast to long regressions that consider potential omitted variables.} to estimate the effect of kindness on the difference in effort provision: 
%
%\begin{align*}
%\Delta \text{\texttt{effort}}_{ij} & = \alpha + \beta X_i + \gamma
%									\text{\texttt{productivity}}_i + 
%									\delta \text{\texttt{info}}_j + 
%									\rho(\text{\texttt{productivity}}_i \times 
%									\text{\texttt{info}}_j) + \eta_i \\
%         						& = \alpha + \beta X_i + \gamma
%									\text{\texttt{productivity}}_i + \delta \text{\texttt{info}}_j 
%									+ \rho\text{\texttt{kindness}}_{ji} + \eta_i 
%\end{align*}
%and 
%\begin{align*}
%\Delta \text{\texttt{nframes}}_{ij} & = \alpha + \beta X_i + \gamma
%									\text{\texttt{productivity}}_i + 
%									\delta \text{\texttt{info}}_j + 
%									\rho(\text{\texttt{productivity}}_i \times 
%									\text{\texttt{info}}_j) + \eta_i \\
%         						& = \alpha + \beta X_i + \gamma
%									\text{\texttt{productivity}}_i + \delta \text{\texttt{info}}_j 
%									+ \rho\text{\texttt{kindness}}_{ji} + \eta_i 
%\end{align*}
%Where we intend to interpret \(\rho\) as the difference in differences mentioned above for an agent \(i\) facing the info decision of her matched principal \(j\).
%
%It is reasonable to assume heterogeneity in the strength of reciprocity. Let this strength be denoted as \(Y_{ij}\), that is, the strength of reciprocity considerations of agent \(i\) towards a principal \(j\). High values of  \(Y_{ij}\) would lead agent \(i\) to reward (punish) kindness (unkindness) strongly, while low values would lead her to behave as if she was maximizing her material payoff. The effect, \(\rho\), of \texttt{kindness} on the outcome  would therefore increase in such a parameter. \(Y_{ij}\) leads to biased estimates of \(\rho\) if it correlates with both, \texttt{kindness} \emph{and} \(\Delta\text{\texttt{effort}}\). Remember that \texttt{kindness} is a function of the principal's choice and the agent's productivity that is, the agent's effort provision in stage 1. Since the roles are assigned randomly and anonymously, there is no reason to assume \(Y_{ij}\) to be related to the principal's choice. Because the real effort task in stage 1 is an individual task that does not affect any other subjects, the strength of reciprocity preferences should not have any causal relation to \text{productivity}. As long as there is no other channel leading to a correlation between \text{productivity} we can assume that \(Y_{ij}\indep\)(\texttt{info, productivity}) \(\Leftrightarrow Y_{ij}\indep\)\texttt{kindness} and that the existence of heterogeneity in \(\rho_i\) adds some noise to our regression but is not a source of OVB.
%
%Furthermore and as introduced in the first section, agents have some preferences towards managerial attention, irrespective of the principal's \texttt{info}-decision. For instance, some may favor attention due to some belief in a meritocratic system while others perceive it as a lack of trust, which causes some psychological costs. Such preferences could lead agents to decrease (increase) their effort provision even though they have been treated in a kind (unkind) manner by the principal. One could model this using a \emph{latent index model} used in the instrumental variables literature where also the terms of compliers, always takers etc. stem from.\footnote{These models describe individual choices as being determined by a comparison of partly observed and unobserved variables such as \texttt{kindness} and some preferences \(\gamma_i\) respectively.
%\[\Delta\text{\texttt{effort}} 
%	\begin{cases}
%    	\geq 0      & \quad \text{if \texttt{kindness}}_{ji}\cdot Y_{ij} + \gamma_i \geq 0\\
%    	< 0  		& \quad \text{otherwise} 
%  	\end{cases}\]
%If an agent \(i\) has, for instance strong preferences against managerial attention, leading to a highly negative \(\gamma_i\) such that \texttt{kindness}\(_{ji} \cdot Y_{ij} < \gamma_i\), this specific agent would never increase her effort|no matter how kind she perceives she decision of a principle \(j\) or how strong her reciprocity preferences are. % But since we cannot observe most of the variables of such a model it is not of much empirical relevance.
%}
%Referring to the regression model, such, to the productivity unrelated, preferences would be captured by the \texttt{info}-main effect and not by it's interaction with \texttt{productivity}. It would consequently not bias our estimate of \(\rho\).
%
%One way to test our second hypothesis is to run a linear probability model (or non-linear models such as probit and logit or simple non-parametric tests like the Mann-Whitney) that explains the principal's (\(j\)'s) \texttt{info} decision in the light of the agent's (\(i\)'s) productivity and a vector of possible covariates \(X_j\)
%\begin{align*}
%\text{\texttt{info}}_{ji} & = \alpha + \beta X_j +\rho \text{\texttt{ productivity}}_i + \eta_j
%\end{align*}
%where \(\rho\) would be our parameter of interest. The covariates are not necessary and would only be introduced to make the estimate of \(\rho\) sharper. We would reject our hypothesis if the estimate of \(\rho\) was non-positive. 
%
%If it turned out that the principals' decisions were perfectly explained by the agents' initial productivities, we would observe \emph{either} kind \emph{or} unkind behavior, but not both of it in our data set. In other words, the more principals fear the agents' reciprocity, the more kind their decision will be, the more difficult it becomes for us to test whether this fear is justified.



%%%%%%%%%%%%%%%%%%%%%%%%%%%%%%%%%%%%%%%%%%%%%%%%%%%%%%%%%%%%%%%%%%%%%%%%%%%%%%%%%%%%%%%%%%%%%%%%%
%\section{Concerns and Open Questions}
%\begin{enumerate}
%\item If we change the principal's payoff function such that it is independent of the agent's effort provision, agents have no possibility to harm the principal and can thus not exhibit reciprocity. As a consequence, we could test whether reciprocity can be ruled out as a driver of possible effects. However, the principal had no reason to invest in information technology in the first place. We could, however, design two treatments which both leave room for reciprocity but differ such that the principals' payoffs are more or less affected by the agent's effort. For instance, we could run a treatment with the principal's wage as the matched agent's piece rate and one where the agent's effort increases the principal's probability of receiving a high wage. 

%\item How do we interpret the number of screens chosen? Can every number of screens lower than the maximum unambiguously be interpreted as a deliberate reduction of effort? \(\rightarrow\) This depends on the exact design so how exactly do we design it? Do those, who choose less than the maximum number have more time per screen? A longer time period per screen might induce the trade-off between a lower but more controllable maximal effort provision regardless of the principal's info decision. This might lead agents with a 68\% productivity to choose 7 out of 10 frames even though they want to pass back kindness.

%\item What if it was costless for the principal to receive a high quality signal. Productive agents' will then perceive it as even more unkind if the principal chooses the low quality signal. Facing productive agents and no costs, the principal has, however, no reason to do so.

%\item The effort difference within subjects might be downward biased because people need to complete (parts of) stage 1 to learn how boring r exhausting it is. Even though this would not be much of a problem because we are not interested in the level of this difference, it might be worth introducing a trial period such that the effort provision in stage 1 and 2 are more comparable.

%\item How shall we set the parameters $w, b, c_H, c_L, q \ \&\  p$?

%\end{enumerate}






%%%%%%%%%%%%%%%%%%%%%%%%%%%%%%%%%%%%%%%%%%%%%%%%%%%%%%%%%%%%%%%%%%%%%%%%%%%%%%%%%%%%%%%%%%%%%%%%%
%%%%%%%%%%%%%%%%%%%%%%%%%%%%%%%%%%%%%%%%%%%%%%%%%%%%%%%%%%%%%%%%%%%%%%%%%%%%%%%%%%%%%%%%%%%%%%%%%
%%%%%%%%%%%%%%%%%%%%%%%%%%%%%%%%%%%%%%%%%%%%%%%%%%%%%%%%%%%%%%%%%%%%%%%%%%%%%%%%%%%%%%%%%%%%%%%%%


\newpage 
\bibliographystyle{apalike} %IEEEtran %apacite %abbrv
\bibliography{Master_Thesis}
\end{document}

